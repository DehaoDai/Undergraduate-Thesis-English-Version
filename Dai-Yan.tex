% !TeX spellcheck = <none>
\documentclass[11pt]{amsart}
\usepackage{float}
\usepackage{amsfonts,amsmath,amssymb}
\usepackage{cases}
\usepackage{mathrsfs,fancyhdr,dsfont,bbm}
\usepackage{anysize}
\usepackage{graphicx}
\usepackage{subfigure}
\usepackage{indentfirst}
\usepackage{extarrows}
\usepackage{enumerate}
\usepackage{caption}
\usepackage{mathrsfs}


\pagestyle{plain}
\renewcommand{\baselinestretch}{1.1}\setlength{\topmargin}{-1cm} \setlength{\oddsidemargin}{0.5cm}
\setlength{\evensidemargin}{0.5cm}
\renewcommand{\headrulewidth}{0pt}
\setlength{\textwidth}{15cm}
\setlength{\textheight}{24cm}


\renewcommand{\thefootnote}{\fnsymbol{footnote}}

\theoremstyle{plain}
\newtheorem{theorem}{Theorem}[section]
\newtheorem{prop}{Proposition}[section]
\newtheorem{lemma}{Lemma}[section]
\newtheorem{conjecture}{Conjecture}[section]
\newtheorem{claim}{Claim}[section]
\newtheorem{definition}{Definition}[section]
\newtheorem{remark}{Remark}
\newtheorem{cor}{Corollary}[section]

\numberwithin{equation}{section}


\usepackage{color}
%\usepackage{pstricks}
%\usepackage{pst-node}
\usepackage{graphicx}
\usepackage{epstopdf}
\renewcommand{\refname}{References}

\usepackage{hyperref}

%%%%%%%%%%%%%%%%%%%%%%%%%%%%%%%%%%%%%%%%%%%%
\pagestyle{fancy}
\fancyhf{}
\fancyhead[EL,OR]{\footnotesize \thepage}
\fancyhead[EC]{\footnotesize D. DAI AND L. YAN}
\fancyhead[OC]{\footnotesize\uppercase \title{ Law of Large Numbers for the Linear Self-interacting Diffusion Driven by $\alpha$-Stable Motion}}
\fancyfoot{}


%%%%%%%%%%%%%%%%%%%%%%%%%%%%%%%%%%%%%%%%%%%%
%%%%%%%%%%%%%%%%%%%%%%%%%%%%%%%%%%%%%%%%%%%%
%%%%%%%%%%%%%%%%%%%%%%%%%%%%%%%%%%%%%%%%%%%%
%%%%%%%%%%%%%%%%%%%%%%%%%%%%%%%%%%%%%%%%%%%%

\begin{document}
\title{Law of Large Numbers for the Linear Self-interacting Diffusion Driven by $\alpha$-Stable Motion}


\footnote[0]{${}^\dag$ddai@ucsd.edu, ${}^{\S}$litan-yan@hotmail.com (Corresponding Author)}

\date{}

\author[D. Dai and L. Yan]{Dehao Dai$^{1,\dag}$ and Litan Yan${}^{2,\S}$}

\keywords{$\alpha$-stable motion, self-interacting diffusion, stable integral, law of large numbers, almost surely convergence, convergence in $L^p$ with $0<p<\alpha$}

\maketitle

\begin{center}
{\footnotesize {\it {${}^1$Department of Mathematics, College of Science, Donghua University, 2999 North Renmin Rd.\\Songjiang, Shanghai 201620, P.R. China\\
${}^2$Department of Statistics, College of Science, Donghua University, 2999 North Renmin Rd.\\Songjiang, Shanghai 201620, P.R. China}}}
\end{center}

\maketitle

%%%%%%%%%%%%%%%%%%%%%%%%%%%%%%%%%%%%%%%%%%%%%%
%%%%%%%%%%%%%%%%%%%%%%%%%%%%%%%%%%%%%%%%%%%%%%
%%%%%%%%%%%%%%%%%%%%%%%%%%%%%%%%%%%%%%%%%%%%%%
%%%%%%%%%%%%%%%%%%%%%%%%%%%%%%%%%%%%%%%%%%%%%%

\begin{abstract}
In this paper, we discuss the linear self-interacting diffusion driven by $\alpha$-stable motion of the form
$$
X_t^\alpha= M_t^\alpha-\theta\int_0^t\int_0^s (X_t^\alpha-X_s^\alpha)ds dt+\nu t,\quad t\geq 0,
$$
where $\theta \neq 0$, $\nu \in \mathbb{R}$ and $M^\alpha_t$ is an $\alpha$-stable motion on $\mathbb{R} (0<\alpha<2)$ with L\'{e}vy symbol $\varphi(u)=|u|^\alpha$. The process is a continuous analogue of the self-attracting diffusion (see M. Cranston and Y. Le Jan\cite{Cranston}). The main purpose of this paper is to introduce some laws of large numbers associated with the process with $1<\alpha<2$. When $\theta>0$, we show that the convergence 
$$
\quad \frac{1}{T}\int_0^T Y_t^\alpha dt\longrightarrow\frac{\nu}{\theta},\quad \frac{1}{T^{4-\frac{2}{\alpha}}}\int_0^T Y_t^\alpha dt\longrightarrow0
$$
hold almost surely and in $L^p$ with $0<p<\alpha$, as $T$ tends to infinty, where $Y^\alpha_t=\int_0^t(X_t^\alpha-X_s^\alpha)ds$. On the other hand, when $\theta<0$, we have $\frac{1}{T}\int_0^T Y_t^\alpha dt\longrightarrow \infty$, as $T$ tends to infinity. Therefore, we show that $\Lambda_t^\alpha: = e^{\frac{1}{2}\theta t^2}Y_t^\alpha$ and $\Xi_t^\alpha :=te^{\frac{1}{2}\theta t^2}\int_0^t Y_t^\alpha dt$ converges almost surely and in $L^p$ with $0<p<\alpha$ to $\xi_\infty^\alpha-\frac{\nu}{\theta}$ and $\frac{1}{\theta}\left(\xi_\infty^\alpha-\frac{\nu}{\theta}\right)$, as $t$ tends to infinity, respectively, where $\xi_\infty^\alpha=\int_0^\infty se^{\frac{1}{2}\theta s^2}dM_s^\alpha$.
\end{abstract}

\section{\textsc{Introduction}}
In 1992, R. Durrett and L.C.G Rogers introduced a stochastic differential model for the shape of a growing polymer\cite{Durrett}. Under some conditions, they established the asymptotic behavior of solution of stochastic differential equation with the dependent path:
\begin{equation}\label{sec1-eq1}
X_t = B_t +\int_0^t \int_0^s f(X_s-X_u)duds
\end{equation}
where $B_t$ is a $d$-dimensional standard Brownian motion and $f$ is Lipschitz continuous. If $f(x)=g(x)x/ \|x\|$ and $g(x) \geq 0$, the solution $X_t$ is a continous analogue of a discrete process introduced by Diaconis and studied by Pemantle\cite{Pemantle}. The solution process $\{X_t\}$ corresponds to the end postion of the polymer at time $t$. Let $\mathscr{L}^X(t,x)$ be the local time of the solution process $X$. Therefore, we have 
$$
X_t = X_0 + B_t+\int_0^t ds \int_\mathbb{R}f(-x)\mathscr{L}^X(s,X_s+x)dx.
$$
for all $t\geq 0$. We may call this solution a Brownian motion interacting with its own passed trajectory, i.e., a \emph{self-interacting motion}. In 1995, Cranston and Le Jan\cite{Cranston} discussed self-attracting diffusions, and when $d=1$ the convergence of the following two cases are studied:
\begin{enumerate}[(i)]
  \item the linear interaction, i.e.$ f(x) = \theta x $, namely,
  \begin{equation*}
X_t=B_t-\theta\int_0^t\int_0^s(X_s-X_u)duds,\quad t\geq 0,
  \end{equation*}
  with $\theta>0$;
  \item the constant interaction, i.e.$f(x) = \sigma \textbf{sgn}(x)$, where $\sigma >0$, namely,
  \begin{equation*}
X_t=B_t-\sigma\int_0^t\int_0^s \textbf{sgn}(X_s-X_u)duds,\quad t\geq 0.
  \end{equation*}
\end{enumerate}
In general, the equation (\ref{sec1-eq1}) defines a self-interacting diffusion without any assumption on $f$. If $x \cdot f(x) \geq 0$(resp. $\leq 0$) for all $x \in \mathbb{R}^d$, in other words if it is more likely to stay away from (resp. stay close to) the position it has already reached before, we could describe it self-repelling (resp. self-attracting). In 2002, Bena\"{i}m \emph{ et al}\cite{Benaim} studied a self-interacting diffusions depending on the (convoluted) empirical measure $\{\mu_t,t\geq 0 \}$, as follows
$$
dX_t = \sqrt{2}dB_t - \bigg(\frac{1}{t}\int_0^t \nabla W(X_t - X_s)ds\bigg)dt,
$$
where $W$ is an interacting potential function. A great difference between these diffusions or Brownian polymers is whether the drift term is divided by $t$. More extended works can be refered in Bena\"{i}m \emph{et al.}\cite{Benaim}, Chambeu and Kurtzmann\cite{Chambeu}, Cranston and Mountford\cite{Cranston2}, Gan and Yan \cite{Gan}, Gauthier\cite{Gauthier}, Herrmann and Roynette\cite{Herrmann}, Herrmann and Scheutzow\cite{Herrmann2}, Kleptsyny and Kurtzmann\cite{Kleptsyny}, Mountford and Tarr\cite{Mountford}, Toth and Werner\cite{Werner}, Sun and Yan \emph{et al.}\cite{Yan,Yan2} and other references.

As a natural extension inspired by Sun and Yan\cite{Yan2}, one can consider the stochastic differential equation driven by $\alpha$-stable motion of the form
\begin{equation}\label{sec1-eq2}
X_t^\alpha = M_t^\alpha + \int_0^t\int_0^s f(X_s^\alpha - X_r^\alpha) dr ds,
\end{equation}
with $X_0^\alpha = 0$, where $M_t^\alpha$ is an $\alpha$-stable L\'{e}vy process on $\mathbb{R}^d$ and $f$ is a Borel measurable function. It is simple to show that the equation mentioned above admits a unique strong solution if $f$ is Lipschitz continuous. On the other hand, for $f$ locally bounded, (\ref{sec1-eq2}) admits a unique weak solution. The solution is called the self-repelling (resp. self-attracting) diffusion driven by an $\alpha$-stable L\'{e}vy process, if $x \cdot f(x) \geq 0$ (resp. $\leq 0$) for all $x \in \mathbb{R}^d$.

In this paper, we consider the law of large numbers for the linear case with jumps
\begin{equation}\label{sec1-eq3}
X_t^\alpha = M_t^\alpha -\theta \int_0^t \int_0^s (X_s^\alpha -X_t^\alpha)drds+\nu t,
\end{equation}
where $M_t^\alpha$ is an $\alpha$-stable motion (a strictly symmetric $\alpha$-stable L\'{e}vy motion) on $\mathbb{R} (0<\alpha<2)$, $\theta \neq 0$ and $\nu \in \mathbb{R}$. Some related interesting questions are considered in the future papers such as parameter estimations and their asymptotic behaviors. 

The structure of this paper includes three sections. In Section 2, we briefly recall $\alpha$-stable L\'{e}vy process and It\^{o}-type stochastic integral with respect to it. In Section 3, we consider the case $\theta>0$. When $1<\alpha<2$ and $M^\alpha$ have no postive jumps, we show that there exists $\kappa$ depending only on $\alpha$ such that
$$
\limsup_{t\rightarrow \infty}\frac{1}{(t\log t)^{1-\frac{1}{\alpha}}}\int_0^t \left(X_t^\alpha-X_s^\alpha\right)ds=\kappa\qquad \text{a.s.}
$$
and the convergence (law of large numbers)
$$
\frac{1}{T}\int_0^T Y_t^\alpha dt\longrightarrow\frac{\nu}{\theta}
$$
almost surely and in $L^p$ with $0<p<\alpha$, as $T$ tends to infinty, where $Y^\alpha_t=\int_0^t(X_t^\alpha-X_s^\alpha)ds$.

In Section 4, we consider the case $\theta<0$. However, when $\theta<0$, we have
$$
\frac{1}{T}\int_0^T Y_t^\alpha dt\longrightarrow \infty
$$
in probability, as $T$ tends to infinity. Therefore, for $\frac{1}{2}<\alpha<2$, we show that the following convergence hold: 
When $\theta<0$, we have
\begin{equation*}
\begin{aligned}
&\Lambda_t^\alpha: = e^{\frac{1}{2}\theta t^2}Y_t^\alpha \longrightarrow \xi_\infty^\alpha-\frac{\nu}{\theta}\\
&\Xi_t^\alpha :=te^{\frac{1}{2}\theta t^2}\int_0^t Y_t^\alpha dt\longrightarrow \frac{1}{\theta}\left(\xi_\infty^\alpha-\frac{\nu}{\theta}\right)
\end{aligned}
\end{equation*}
almost surely and in $L^p$ with $0<p<\alpha$, where $\xi_\infty^\alpha=\int_0^\infty se^{\frac{1}{2}\theta s^2}dM_s^\alpha$.

\section{\textsc{Preliminaries}}
In this section, we will recall the definition and some basic facts of $\alpha$-stable motion and the definition and properties of It\^{o}-type stochastic integral for $\alpha$-stable motion on $\mathbb{R}$. Concerning more notions and background, we can refer to Rosinski and Woyczynski\cite{Rosinski}, Kallenberg\cite{Kallenberg}, Applebaum\cite{Applebaum}, Samorodnitsky and Taqqu\cite{Samorodnitsky}, and Sato\cite{Sato}.

\subsection{$\alpha$-stable L\'{e}vy process}
Throughout this paper we define a complete probability space ($\Omega,\mathcal{F},P,\{\mathscr{F}_t\}$) such that the processes are well-defined on the space. For simplicity we let $C_{(\cdot)}$ be a postive constant depending only on its subscripts and its value may be different under different conditions, and this assumption is also adaptable to $c_{(\cdot)}$.

Assume four parameters $\alpha,\lambda,\beta,\mu$ satisfying 
$$
\alpha \in(0,2],\quad \lambda \in(0,+\infty),\quad \beta\in[-1,1],\quad \mu\in(-\infty,+\infty),
$$
and denote
\begin{equation*}
\phi_\alpha(u) =
\begin{cases}
-\lambda^\alpha|u|^\alpha(1-i\beta \text{sgn}(u)\tan\frac{\alpha\pi}{2})+i\mu u, & \alpha \neq 1 ,\\
-\lambda|u|(1+i\beta\frac{2}{\pi} \text{sgn}(u)\log|u|)+i\mu u,&\alpha =1,\\
\end{cases}
\end{equation*}
where $u\in(-\infty,+\infty)$, $i^2=-1$. A random variable $\eta$ subjects to an $\alpha$-stable distribution, denoted by $\eta \sim \mathcal{S}_\alpha(\lambda,\beta,\mu)$, if it has the characteristic form
\begin{equation*}
E(e^{iu\eta})=e^{-\phi_\alpha(u)},
\end{equation*}
These parameters $\alpha,\lambda,\beta,\mu$ are called the stability index, scale one, skewness one, and location one, respectively. When $\mu=0$, we may call $\eta$ to be strictly $\alpha$-stable, and if $\beta=0$ additionally, we may call $\eta$ previously to be symmetrically $\alpha$-stable. 

An $\{\mathscr{F}_t\}$-adapted process $M^\alpha = \{M_t^\alpha, t \geq 0\}$ with all sample paths in $D[0,\infty)$ could be an $\alpha$-stable motion (an $\alpha$-stable L\'{e}vy process) with $\alpha \in (0,2]$, if for any $t>s\geq 0$, we have 
$$
E\left[e^{-iu(M^\alpha_t-M^\alpha_s)}|\mathscr{F}_s\right]=e^{-(t-s)\phi _\alpha(u)},\quad  u\in \mathbb{R},
$$
where $\phi _\alpha(u)$ is called the L\'{e}vy symbol of $M^\alpha$. In this paper, we assume that the $\alpha$-stable motion $M^\alpha$ is strictly symmetric (namely $\mu=\beta=0$ and $\lambda=1$), i.e. $\phi_\alpha(u)=|u|^\alpha$.

\subsection{It\^{o}-type stochastic integral for $\alpha$-stable L\'{e}vy process}
In this part, we recall the definition and properties of It\^{o}-type stochastic integral for $\alpha$-stable L\'{e}vy process\cite{Protter}
$$
A_t^\alpha = \int_0^t F_s dM_s^\alpha, \quad t\geq 0,
$$
where $F=\{F(t,\omega)\}_{t\geq 0}$ is an adapted $\{\mathscr{F}_t\}$ process on $\Omega[0,\infty)$ and $M^\alpha$ is an $\alpha$-stable motion (strictly symmetric, $\lambda=1$). If for every $T>0$, we have 
$$
F\in L^\alpha_{a.s} = \left\{F: \{\mathscr{F}_t\}-\text{adapted}|\int_0^T|F_t|^\alpha dt<\infty,a.s.\forall T>0 \right\},
$$
almost surely, the stable stochastic integral exists for $\alpha \in (0,2)$. In general, $F=\{F(t,w)\}_{t\geq 0}$ has the form as follows
$$
F(t,\omega):=\varphi_0(\omega)\mathbbm{1}_{\{t=0\}}(t)+\sum_{i=0}^\infty\varphi_i(\omega)\mathbbm{1}_{(t_i,t_{i+1}]}(t),
$$
where $0=t_0<t_1<\cdots<t_n<\cdots\rightarrow\infty$ and $\varphi_n$ is $\mathscr{F}_{t_n}$ measurable for every $n=0,1,2\cdots$. The stochastic integral with respect to $F$ could be defined:
\begin{equation*}
\begin{aligned}
\int_0^tF(s,\omega)dM^\alpha(s,\omega)=&\sum_{i=0}^{n-1}\varphi_i(\omega)(M^\alpha(t_{i+1},\omega)-M^\alpha(t_i,\omega))\\
&+\varphi_n(\omega)(M^\alpha(t,\omega)-M^\alpha(t_n,\omega))
\end{aligned}
\end{equation*}
where $t_n\leq t\leq t_{n+1},n=0,1,2\cdots$. Moreover, the following estimates holds:
\begin{equation}\label{sec2-eq1}
c_\alpha\int_0^T E[|F_s|^\alpha]ds\leq\sup_{\lambda>0}\lambda^\alpha P\left(\sup_{0\leq t \leq T} \left|\int_0^t F_sdM_s^\alpha\right|\geq \lambda\right)\leq C_\alpha\int_0^T E[|F_s|^\alpha]ds
\end{equation}
for all $T\geq 0$. The left-hand side of the inequalities is established in Rosinski-Woyczynski\cite{Rosinski} and the right-hand side of the ones is considered in Gin\'{e}-Marcus\cite{Gine}

\begin{theorem}[Rosinski-Woyczynski\cite{Rosinski}]
Let $0<\alpha<2$ and for any $F\in L^\alpha_{a.s}$, we define
$$
\tau_\alpha(u) =\int_0^u |F_t|^\alpha dt.
$$
such that $\tau_\alpha(u)$ tends to infinity almost surely, as $u$ tends to infinity. Denote $\tau_\alpha^{-1}(t)=\inf\{u:\tau_\alpha(u)>t\}$ and $\mathscr{A}_t=\mathscr{F}_{\tau_\alpha^{-1}(t)}$. The time-changed stochastic integral
$$
\widetilde{M}_t^\alpha=\int_0^{\tau_\alpha^{-1}(t)}F_sdM_s^\alpha
$$
is an $\{\mathscr{F}_t\}$-$\alpha$ stable motion on $\mathbb{R}$. Consequently, for each $t>0$, we have 
$$
\widetilde{M}^\alpha_{\tau_\alpha(t)}=\int_0^tF_sdM_s^\alpha
$$
almost surely. 
\end{theorem}

\section{\textsc{Law of large numbers, self-attracting case}}
Throughout this paper, we assume that $M^\alpha$ is a strictly symmetric $\alpha$-stable motion with $\alpha \in (0,2)$, $\theta \neq 0, \nu \in \mathbb{R}$. In this section, we consider the convergence of the solution of the equation
\begin{equation}\label{sec3-eq1}
X_t^\alpha = M_t^\alpha -\theta \int_0^t\int_0^s (X_s^\alpha - X_r^\alpha) dr ds +\nu t,
\end{equation}
where $\theta > 0,\nu \in \mathbb{R}$ and $M^\alpha$ is an $\alpha$-stable motion on $\mathbb{R}$ ($0<\alpha<2$) with $M^\alpha_0=0$.

We introduce the kernel function $(t,s)\mapsto h_\theta (t,s)$ by
\begin{equation}\label{sec3-eq2}
h_\theta(t,s) =
\begin{cases}
1-\theta s e^{\frac{1}{2}\theta s^2}\int_s^t e^{-\frac{1}{2}\theta u^2} du, &\quad t \geq s,\\
0 , &\quad t<s,\\
\end{cases}
\end{equation}
for any $s,t \geq 0$. The kernel function $(t,s)\mapsto h_\theta (t,s)$ has the following properties

(1)When $\theta >0$, the limit
\begin{equation*}
h_\theta (s):= \lim_{t\rightarrow \infty}h_\theta(t,s)=1-\theta se^{\frac{1}{2}\theta s^2}\int_s^\infty e^{-\frac{1}{2}\theta u^2}du
\end{equation*}
exists for all $s \geq 0$; when $\theta <0$, we have the limit
\begin{equation}
\lim_{t\rightarrow \infty} \left(te^{\frac{1}{2}\theta t^2}h_\theta(t,s)\right)=se^{\frac{1}{2}\theta s^2}
\end{equation}
for any $s \geq 0$.

(2) When $\theta >0$, we have $h_\theta (s) \leq h_\theta (t,s)$ and 
\begin{equation}
e^{-\frac{1}{2}\theta(t^2-s^2)}\leq h_\theta (t,s) \leq 1
\end{equation}
for all $t\geq s\geq 0$. Moreover, we also have 
\begin{equation*}
0 \leq h_\theta (s) \leq  C_\theta \min \left\{1,\frac{1}{s^2}\right\}
\end{equation*}
for all $s\geq 0$.

(3) When $\theta <0$, we have
\begin{equation*}
1 \leq h_\theta (t,s) \leq e^{-\frac{1}{2}\theta(t^2-s^2)} 
\end{equation*}
for all $t \geq s \geq 0$.

(4) When $\theta \neq 0$, we have 
\begin{equation*}
h_\theta (t,0)=h_\theta (t,t)=1,\quad \int_u^t h_\theta (t,s)ds = e^{\frac{1}{2}\theta u^2 }\int_u^t  e^{-\frac{1}{2}\theta s^2} ds.
\end{equation*}
for all $t \geq u \geq 0$.

Using the kernel function $(t,s)\mapsto h_\theta (t,s)$, we can simply show that the solution of (\ref{sec3-eq1}) admits the following representation form when $t \geq 0$:
\begin{equation}\label{sec3-eq9}
X_t^\alpha = \int_0^t h_\theta(t,s)dM^\alpha_s+\nu \int_0^t h_\theta (t,s)ds.
\end{equation}
By the method of constant variation (see Cranston and Le Jan\cite{Cranston}), we could introduce the above representation. Spontaneously, we could also prove this through the integration by part. Therefore, we only need to check that the process $X^\alpha$ defined by (\ref{sec3-eq9}) is the solution of (\ref{sec3-eq1}), because it is obvious to show that the solution (\ref{sec3-eq1}) is unique. Noting that
\begin{equation*}
\begin{aligned}
X_t^\alpha &= \int_0^t h_\theta(t,s)dM^\alpha_s+\nu \int_0^t h_\theta (t,s)ds\\
& = \int_0^t \left(1-\theta s e^{\frac{1}{2}\theta s^2}\int_s^t e^{-\frac{1}{2}\theta u^2} du\right)dM_s^\alpha+\nu \int_0^t e^{-\frac{1}{2}\theta u^2}du\\
& = M_t^\alpha - \theta \int_0^t se^{\frac{1}{2}\theta s^2}dM_s^\alpha\int_s^t e^{-\frac{1}{2}\theta u^2}du+\nu \int_0^t h_\theta (t,s)ds\\
& = M_t^\alpha - \int_0^t e^{-\frac{1}{2}\theta u^2} \left(\int_0^u \theta se^{\frac{1}{2}\theta s^2}dM_s^\alpha \right)du+ \nu\int_0^t e^{-\frac{1}{2}\theta u^2}du \\
& = M_t^\alpha - \int_0^t e^{-\frac{1}{2}\theta u^2} \left(\int_0^u \theta se^{\frac{1}{2}\theta s^2}dM_s^\alpha -\nu \right)du
\end{aligned}
\end{equation*}
for all $t \geq 0$, we see that the process $X_\alpha$ defined by (\ref{sec3-eq9}) is the solution of (\ref{sec3-eq1}) if and only if for all $ t \geq u \geq 0$, we have 
\begin{equation}\label{sec2-eq12}
Y_u^\alpha: = \int_0^u (X_u^\alpha - X_s^\alpha)ds =  e^{-\frac{1}{2}\theta u^2} \left(\int_0^u se^{\frac{1}{2}\theta s^2}dM_s^\alpha \right) +\frac{\nu}{\theta}\left(1-e^{-\frac{1}{2}\theta u^2}\right)
\end{equation}
In fact, through the integration by part
\begin{equation*}
\int_0^t s dM_s^\alpha = tM_t^\alpha - \int_0^t M_s^\alpha ds,
\end{equation*}
for any $t \geq 0$ and $\alpha \in (0,2]$, we have that 
\begin{equation*}
\begin{aligned}
& Y_u =\int_0^u (X_u^\alpha - X_s^\alpha)ds = u X_u^\alpha -  \int_0^u X_s^\alpha ds\\
& = u X_u^\alpha -\int_0^u \left(\int_0^s h_\theta (s,r) dM_r^\alpha + \nu \int_0^s h_\theta (s,r) dr \right)ds\\
& =  u X_u^\alpha - \int_0^u \left(M_s^\alpha - \int_0^s \theta r e^{\frac{1}{2}\theta r^2 }\left(\int_r^s e^{-\frac{1}{2}\theta x^2 }dx \right) dM_r^\alpha +\nu \int_0^s e^{-\frac{1}{2}\theta x^2 }dx\right)ds\\
& =  u X_u^\alpha -\int_0^u M_s^\alpha ds\\
& \quad \quad +\int_0^u ds \int_0^s \theta r e^{\frac{1}{2}\theta r^2}\left(\int_r^s e^{-\frac{1}{2}\theta x^2}dx \right) dM_r^\alpha - \nu \int_0^u ds \int_0^s e^{-\frac{1}{2}\theta x^2}dx\\
& =  uX_u^\alpha - uM_u^\alpha + \int_0^u sdM_s^\alpha\\
& \quad \quad +\int_0^u e^{-\frac{1}{2}\theta x^2}\left(\int_0^x \theta r e^{\frac{1}{2}\theta r^2 }dM_r^\alpha \right)(u-x)dx -\nu \int_0^u e^{-\frac{1}{2}\theta x^2}(u-x)dx\\
& = \int_0^u s dM_s^\alpha +u \left[X_u^\alpha -M_u^\alpha + \int_0^u e^{-\frac{1}{2}\theta x^2}\left(\int_0^x \theta r e^{\frac{1}{2}\theta r^2}dM_r^\alpha \right)dx -\nu \int_0^u e^{-\frac{1}{2}\theta x^2}dx\right]\\
& \quad \quad - \int_0^u x e^{-\frac{1}{2}\theta x^2} \left(\int_0^x \theta r e^{\frac{1}{2}\theta r^2}dM_r^\alpha \right)dx + \frac{\nu}{\theta} \left(1 - e^{-\frac{1}{2}\theta u^2}\right)\\
& = \int_0^u s dM_s^\alpha - \int_0^u x e^{-\frac{1}{2}\theta x^2} \left(\int_0^x \theta r e^{\frac{1}{2}\theta r^2}dM_r^\alpha \right)dx + \frac{\nu}{\theta} \left(1 - e^{-\frac{1}{2}\theta u^2}\right) \\
& = \int_0^u s dM_s^\alpha  - \int_0^u r e^{\frac{1}{2}\theta r^2}dM_r^\alpha \int_r^u \theta x e^{-\frac{1}{2}\theta x^2} dx + \frac{\nu}{\theta}\left(1-e^{-\frac{1}{2}\theta u^2} \right)\\
& = e^{-\frac{1}{2}\theta u^2} \left(\int_0^u r e^{\frac{1}{2}\theta r^2}dM_r^\alpha  \right)+\frac{\nu}{\theta}\left(1-e^{-\frac{1}{2}\theta u^2} \right) \\
\end{aligned}
\end{equation*}

For all $t \geq u \geq 0$, the formula (\ref{sec2-eq12}) needs to be checked as follows.
\begin{lemma}[(X. Sun and L. Yan\cite{Yan2})]
Let $F=\{F_s,s\geq 0 \}$ be a continuous adapted process belonging to $L^\alpha$. Define the process A by
\begin{equation*}
A_t = \int_{1/t}^\infty |F_s|^\alpha ds
\end{equation*}
with $A_0=0$. Let $\mathscr{F}_t'=\sigma(M_s^\alpha,s \geq \frac{1}{t})$, where$\mathscr{F}_0'=\{\emptyset,\Omega\}$. If $A_t^{-1}=\inf\{u:A_u>t\}$ and $\mathscr{A}_t=\mathscr{F}_{\mathscr{A}^{-1}_t}'$, the time-changed stochastic integral
$$
\widetilde{M}_t^\alpha=\int_{1/A_t^{-1}}^\infty F_sdM_s^\alpha
$$
is an $\{\mathscr{A}_t\}$-$\alpha$-stable motion. Therefore, for each $t>0$, we have 
\begin{equation*}
N_t:=\int_{1/t}^\infty F_sdM_s^\alpha=\widetilde{M}_{A_t}^\alpha,\quad t>0
\end{equation*}
almost surely, with $N_0=0$.
\end{lemma}

\begin{proof}
Similar to Rosinski and Woyczynski\cite{Rosinski}, one can prove the lemma.
\end{proof}

As a corollary of Lemma 3.1, we have
\begin{equation}\label{sec3-eq12}
c_\alpha\int_T^\infty E[|F_s|^\alpha]ds\leq\sup_{\lambda>0}\lambda^\alpha P\left(\sup_{T\leq t \leq \infty} \left|\int_0^t F_sdM_s^\alpha\right|\geq \lambda\right)\leq C_\alpha\int_T^\infty E[|F_s|^\alpha]ds
\end{equation}
for all $T>0$ and $0<\alpha<2$.

\begin{lemma}
Let $\alpha>1$ and $\alpha$-stable motion $M^\alpha$ have no postive jumps. Therefore, there exists $\kappa>0$ depending only on $\alpha$ and $\theta$, such that
\begin{equation}\label{sec3-eq13}
\limsup_{t\rightarrow \infty}\frac{1}{(t\log t)^{1-\frac{1}{\alpha}}}\int_0^t \left(X_t^\alpha-X_s^\alpha\right)ds=\kappa
\end{equation}
almost surely.
\end{lemma}

\begin{proof}
Denote $Y_t^\alpha:=\int_0^t \left(X_t^\alpha-X_s^\alpha\right)ds$. We can get the equivalent decomposition of $Y_t^\alpha$ by the formula (\ref{sec2-eq12}):
\begin{equation*}
Y_t^\alpha =  e^{-\frac{1}{2}\theta t^2} \left(\int_0^t se^{\frac{1}{2}\theta s^2}dM_s^\alpha \right) +\frac{\nu}{\theta}\left(1-e^{-\frac{1}{2}\theta t^2}\right)=\Lambda_1+\Lambda_2.
\end{equation*}
The lemma can be proven in two steps.
\textbf{(I)} Let us study the convergence in probability of $\Lambda_1$. When $t>0$, we could denote
$$
\tau_\alpha(t):=\int_0^t s^\alpha e^{\frac{1}{2}\theta \alpha s^2 }ds.
$$
Theorem 2.1 admits that the $\alpha$-stable motion
$$
\widetilde{M}_t^\alpha:=\int_0^{\tau_\alpha^{-1}(t)}se^{\frac{1}{2}\theta s^2}dM_s^\alpha,\quad t\geq 0
$$
has no postive jumps. Therefore, we have
$$
\widetilde{M}_{\tau_\alpha(t)}^\alpha:=\int_0^t se^{\frac{1}{2}\theta s^2}dM_s^\alpha,\quad t\geq 0.
$$
almost surely, for each $t>0$. Applying the law of iterated logarithm\cite{Breiman}: there exists some positive constant $c>0$ such that 
\begin{equation}\label{sec3-eq14}
\limsup_{t\rightarrow\infty}\frac{M_t^\alpha}{t^{\frac{1}{\alpha}}(\log\log t)^{1-\frac{1}{\alpha}}}=c
\end{equation}
as $t$ tends to infinty, we can obtain
\begin{equation*}
\begin{aligned}
&\quad\limsup_{t\rightarrow \infty}\frac{1}{(t\log t)^{1-\frac{1}{\alpha}}}\Lambda_1 =\limsup_{t\rightarrow\infty}\frac{e^{-\frac{1}{2}\theta t^2}}{(t\log t)^{1-\frac{1}{\alpha}}}\left(\int_0^t se^{\frac{1}{2}\theta s^2}dM_s^\alpha\right)\\
&=c \limsup_{t\rightarrow \infty}\frac{\tau_\alpha(t)^{\frac{1}{\alpha}}(\log\log \tau_\alpha(t))^{1-\frac{1}{\alpha}}}{(t\log t)^{1-\frac{1}{\alpha}}e^{\frac{1}{2}\theta t^2}}\\
& =c \limsup_{t\rightarrow\infty}\left(\frac{\tau_\alpha(t)}{t^{\alpha-1}e^{\frac{1}{2}\alpha\theta t^2}}\right)^{\frac{1}{\alpha}}\limsup_{t\rightarrow\infty}\left(\frac{\log\log \tau_\alpha(t)}{\log t}\right)^{1-\frac{1}{\alpha}}\\
& \xlongequal{\text{L'H\^{o}pital's rule}} c\limsup_{t\rightarrow\infty}\left(\frac{t^\alpha e^{\frac{1}{2}\theta \alpha t^2}}{\alpha \theta t^\alpha e^{\frac{1}{2}\theta \alpha t^2 }}\right)^{\frac{1}{\alpha}}\limsup_{t\rightarrow\infty}\left(\frac{t^{\alpha+1} e^{\frac{1}{2}\theta \alpha t^2}}{\log \tau_\alpha (t)\tau_\alpha(t)}\right)^{1-\frac{1}{\alpha}}\\
&=c\left(\frac{1}{\alpha\theta}\right)^{\frac{1}{\alpha}}\limsup_{t\rightarrow\infty}\left(\frac{\alpha \theta t^2}{1+\log \tau_\alpha(t)}\right)^{1-\frac{1}{\alpha}}\\
& = c\left(\frac{1}{\alpha\theta}\right)^{\frac{1}{\alpha}}2^{1-\frac{1}{\alpha}},
\end{aligned}
\end{equation*}

\textbf{(II)} When $t>0$, we have the following limit:
\begin{equation*}
\begin{aligned}
&\quad\limsup_{t\rightarrow \infty}\frac{1}{(t\log t)^{1-\frac{1}{\alpha}}}\Lambda_2 =\limsup_{t\rightarrow \infty}\frac{\frac{\nu}{\theta}}{(t\log t)^{1-\frac{1}{\alpha}}}\left(1-e^{-\frac{1}{2}\theta t^2}\right)\\
&=-\frac{\nu}{\theta}\limsup_{t\rightarrow\infty}\frac{1}{e^{\frac{1}{2}\theta t^2}(t\log t)^{1-\frac{1}{\alpha}}}=0.
\end{aligned}
\end{equation*}

Thus, we have proved the Lemma by the convergence mentioned above.
\end{proof}

\begin{lemma}
Let $1<\alpha<2$, there exists $\beta>\frac{1}{\alpha}$ (in particular, when $\alpha=2$, we can denote $\beta=1$) such that
\begin{equation}\label{sec3-eq15}
\frac{M_T^\alpha}{T^\beta}\longrightarrow 0\qquad\text{a.s.}
\end{equation}
as t tends to infinity.
\end{lemma}

\begin{proof}
This is a simple exercise. Applying the estimate (\ref{sec2-eq1}) and the fact $M_t^\alpha : = \int_0^t 1dM_s^\alpha$, we can obtain that for any $\varepsilon>0$,
\begin{equation*}
\begin{aligned}
P\left(\sup_{t\geq T}\left|\frac{M_t^\alpha}{t^\beta}\right|\geq \varepsilon\right)&\leq P\left(\frac{1}{T^\beta}\sup_{t\geq T}\left|\int_0^t 1dM_s^\alpha\right|\geq \varepsilon\right) \leq C_\alpha\varepsilon^{-\alpha}\left(\frac{1}{T^\beta}\right)^\alpha\left(\int_0^T 1^\alpha ds\right)\\
&\sim C_{\alpha}\varepsilon^{-\alpha}\frac{1}{T^{\alpha\beta-1}}\longrightarrow 0\quad(T\rightarrow\infty)
\end{aligned}
\end{equation*}
as $t$ tends to infinity.
\end{proof}

\begin{lemma}
Let $1<\alpha<2$, $\theta>0$ and $\alpha$-stable motion $M^\alpha$ have no postive jumps. When $t>0$, we have
\begin{equation}\label{sec3-eq16}
\frac{1}{t^{3-\frac{2}{\alpha}}}Y_t^\alpha =\frac{1}{t^{3-\frac{2}{\alpha}}} \int_0^t (X_u-X_t)du\longrightarrow 0
\end{equation}
almost surely, as $t$ tends to infinity.
\end{lemma}
\begin{proof}
For all $t\geq 0$, denoting $\Delta_t=\frac{\nu}{\theta}\left(1-e^{-\frac{1}{2}\theta t^2}\right)$ and
$$
\eta_t=e^{-\frac{1}{2}\theta t^2}\int_0^t se^{\frac{1}{2}\theta s^2}dM_s^\alpha,
$$
we have $Y_t^\alpha=\eta_t+\Delta_t$ and
\begin{equation*}
\begin{aligned}
\frac{1}{t^{3-\frac{2}{\alpha}}}Y_t^\alpha=\frac{1}{t^{3-\frac{2}{\alpha}}}(\eta_t+\Delta_t)=\frac{1}{t^{3-\frac{2}{\alpha}}}\eta_t+\frac{1}{t^{3-\frac{2}{\alpha}}}\Delta_t:=A_{1t}+A_{2t}.
\end{aligned}
\end{equation*}
For one thing, applying the L'H\^{o}pital rule and the estimate (\ref{sec2-eq1}), we can show that $A_{1t}$ converges to zero almost surely, as $t$ tends to infinity. For any $\varepsilon>0$, we have 
\begin{equation*}
\begin{aligned}
P\left(\sup_{t\geq T}\left|A_{1t}\right|\geq \varepsilon\right)&=P\left(\sup_{t\geq T}\left|\frac{1}{t^{3-\frac{2}{\alpha}}}e^{-\frac{1}{2}\theta t^2}\int_0^tse^{\frac{1}{2}\theta s^2}dM_s^\alpha\right|\geq \varepsilon\right)\\
&\leq P\left(\frac{1}{T^{3-\frac{2}{\alpha}}}e^{-\frac{1}{2}\theta T^2}\sup_{t\geq T}\left|\int_0^t se^{\frac{1}{2}\theta s^2}dM_s^\alpha\right|\geq \varepsilon\right)\\
&\leq C_\alpha \varepsilon^{-\alpha}\left(\frac{1}{T^{3-\frac{2}{\alpha}}e^{\frac{1}{2}\theta T^2}}\right)^\alpha\left(\int_0^T s^\alpha e^{\frac{1}{2}\alpha\theta s^2}ds\right)\\
&\sim C_\alpha\varepsilon^{-\alpha}\frac{1}{\alpha\theta T^{3\alpha-1}e^{\frac{1}{2}\alpha\theta T^2}}\left(T^\alpha e^{\frac{1}{2}\alpha\theta T^2}\right)\\
&=C_\alpha\varepsilon^{-\alpha}\frac{1}{\alpha\theta T^{2\alpha-1}}\longrightarrow0\quad(T\rightarrow\infty)\\
\end{aligned}
\end{equation*}
For another thing, applying the estimate of $\Delta_t$ in Lemma 3.2, we can have 
\begin{equation*}
\begin{aligned}
\lim_{t\rightarrow \infty}A_{2t}=\lim_{t\rightarrow \infty}\frac{1}{t^{3-\frac{2}{\alpha}}}\Delta_t=\lim_{t\rightarrow\infty}\frac{1}{t^{3-\frac{2}{\alpha}}}\frac{\nu}{\theta}\left(1-e^{-\frac{1}{2}\theta t^2}\right)=0
\end{aligned}
\end{equation*}
as $t$ tends to infinity. 

Thus, we have proved the Lemma by the almost surely convergence of $A_{1t}$ and $A_{2t}$.
\end{proof}

\begin{theorem}
Let $1<\alpha<2$ and $\theta>0$. When $T>0$, we have
\begin{equation}\label{sec3-eq17}
\frac{1}{T}\int_0^T Y_t^\alpha dt\longrightarrow\frac{\nu}{\theta}
\end{equation} 
almost surely and in $L^p(0<p<\alpha)$, as $T$ tends to infinity.
\end{theorem}

\begin{proof}
By the solution (\ref{sec3-eq9}) and (\ref{sec1-eq3}), for any $T \geq 0 $, we have 
\begin{equation*}
\begin{aligned}
\left|\frac{1}{T}\int_0^T Y_t^\alpha dt-\frac{\nu}{\theta}\right|&=\frac{1}{\theta }\left|\frac{M_T^\alpha}{T}-\frac{X_T^\alpha}{T}\right|\\
&=\frac{1}{\theta}\left|\frac{M_T^\alpha}{T}-\frac{1}{T}\left(\int_0^T h_\theta(T,t)dM_t^\alpha+\nu\int_0^T h_\theta(T,t)dt\right)\right|\\
&\leq \frac{1}{\theta }\left|\frac{M_T^\alpha}{T}\right|+\frac{1}{\theta}\left|\frac{1}{T}\int_0^T h_\theta(T,t)dM_t^\alpha\right|+\frac{\nu}{\theta}\left|\frac{1}{T}\int_0^T h_\theta(T,t)dt\right|\\
&:=B_{1T}+B_{2T}+B_{3T}
\end{aligned}
\end{equation*}
where it is obvious to show that $B_{1T}=\frac{1}{\theta }\left|\frac{M_T^\alpha}{T}\right|\longrightarrow0(T\rightarrow\infty)$ almost surely by the Lemma 3.3.

For one thing, for any $\varepsilon>0$, we can have
\begin{equation*}
\begin{aligned}
&\quad P\left(\sup_{t\geq T}\left|\frac{1}{t}\int_0^t h_\theta(t,s)dM_s^\alpha\right|\geq \varepsilon\right)\leq P\left(\frac{1}{T}\sup_{t\geq T}\left|\int_0^t h_\theta(t,s)dM_s^\alpha\right|\geq \varepsilon\right)\\
&\leq C_\alpha\varepsilon^{-\alpha}\left(\frac{1}{T}\right)^\alpha\left(\int_0^T h_\theta(t,s)^\alpha ds\right)\leq C_\alpha\varepsilon^{-\alpha}\left(\frac{1}{T}\right)^\alpha\left(\int_0^T 1^\alpha ds\right)\\
&=C_\alpha\varepsilon^{-\alpha}\frac{1}{T^{\alpha-1}}\longrightarrow 0 \qquad (T\rightarrow\infty)
\end{aligned}
\end{equation*}

Thus, we have shown that $B_{2T}$ convergences to zero almost surely, as $T$ tends to infinity.

For another thing, we have
\begin{equation*}
\begin{aligned}
B_{3T}=\frac{\nu}{\theta}\left|\frac{1}{T}\int_0^T h_\theta(T,t)dt\right|=\frac{\nu}{\theta}\left|\frac{1}{T}\int_0^T e^{-\frac{1}{2}\theta t^2} dt\right|\sim \frac{\nu}{\theta}e^{-\frac{1}{2}\theta T^2} \longrightarrow 0
\end{aligned}
\end{equation*}

Finally, we can get the convergence (\ref{sec3-eq17}) in probability. Now, we consider the convergence in $L^p(0<p<\alpha)$. It is clear to see that for all $o<p<\alpha$, we have
\begin{equation*}
\begin{aligned}
E|B_{1t}|^p&=E|\frac{1}{\theta t}M_t^\alpha|^p\sim \frac{1}{\theta^p t^p}E\left|\int_0^t 1dM_s^\alpha\right|^p \\
&\leq C_{p,\alpha} \frac{1}{\theta^p t^p}\left (\int_0^t 1^\alpha ds\right)^{\frac{p}{\alpha}}=C_{p,\alpha}\frac{1}{\theta^p t^{p(1-\frac{1}{\alpha})}}\longrightarrow 0 \quad(t\rightarrow\infty).
\end{aligned}
\end{equation*}

Similarly, we have that
\begin{equation*}
\begin{aligned}
E|B_{2t}|^p&=\frac{1}{\theta^p}E\left|\frac{1}{t}\int_0^t h_\theta(t,s)dM_s^\alpha\right|^p\sim \frac{1}{\theta^p t^p}E\left(\int_0^t h_\theta(t,s)dM_s^\alpha\right)^p\\
&\leq \frac{1}{\theta^p t^p}\left(\int_0^t h^\alpha_\theta(t,s) ds\right)^{\frac{p}{\alpha}}\leq \frac{1}{\theta^p t^{(1-\frac{1}{\alpha})p}}\longrightarrow 0(t\rightarrow \infty)
\end{aligned}
\end{equation*}
converges in $L^p(0<p<\alpha)$, and we can get the Theorem 3.1 by the facts above finally.
\end{proof}

\begin{cor}
Let $1<\alpha<2$ and $\theta>0$, we can have 
\begin{equation}\label{sec3-eq18}
\frac{1}{T^{4-\frac{2}{\alpha}}}\int_0^T Y_t^\alpha dt\longrightarrow0\quad(T\rightarrow\infty)
\end{equation}  
almost surely and in $L^p(0<p<\alpha)$, as $T$ tends to infinity.
\end{cor}

\begin{proof}
This is a simple exercise. It is obvious to show the equation (\ref{sec3-eq18}) in the corollary  by applying the Lemma 3.4 and Theorem 3.1.
\end{proof}

\section{\textsc{Law of large numbers, self-repelling case}}
When $\theta<0$ and $\alpha>0$, we have
$$
\frac{1}{T}\int_0^T Y_t^\alpha dt \longrightarrow \infty
$$
as $T$ tends to infinity, namely, the original law of large numbers is not suitable for this situation. Therefore, in this section, when we discuss the convergence in probability and in $L^p(0<p<\alpha)$ of $Y_t^\alpha$ and $\int_0^t Y_t^\alpha dt$ under the self-repelling case, we need to change the original convergence. Let $X^\alpha$ be the solution of (\ref{sec1-eq3}) with $\theta<0$. We have
$$
X_t^\alpha=\int_0^t h_\theta(t,s)dM_s^\alpha+\nu\int_0^t h_\theta(t,s)ds,
$$
where the kernel function
$$
h_\theta(t,s):=1-\theta se^{\frac{1}{2}\theta s^2}\int_s^t e^{-\frac{1}{2}\theta u^2}du.
$$

\begin{lemma}
Let $\theta<0$ and $0<\alpha\leq 2$. Define the process
$$
\xi_t^\alpha :=\int_0^t se^{\frac{1}{2}\theta s^2}dM_s^\alpha,\quad t\geq 0.
$$
The process $\xi_t^\alpha$ converges to $\xi_\infty^\alpha$ almost surely and in $L^p(0<p<\alpha)$, as $t$ tends to infinity, where the random variable $\xi_\infty^\alpha:=\int_0^\infty se^{\frac{1}{2}\theta s^2}dM_s^\alpha$ is well-defined in $L^p (0<p<\alpha)$.
\end{lemma}

\begin{proof}
Let $\theta<0$ and $0<\alpha\leq 2$. At first, we consider the almost surely convergence as $t$ tends to infinity. It is obvious that we have
\begin{equation*}
\begin{aligned}
\xi_t^\alpha-\xi_\infty^\alpha&=\int_\infty^t se^{\frac{1}{2}\theta s^2}dM_s^\alpha=te^{\frac{1}{2}\theta t^2}M_t^\alpha+\int_t^\infty(1+\theta s^2)e^{\frac{1}{2}\theta s^2}M_s^\alpha ds\\
&:=\zeta_1(t)+\zeta_2(t)
\end{aligned}
\end{equation*}
where based on the fact $M_t^\alpha\sim C_\alpha t^{\frac{1}{\alpha}}(t\rightarrow\infty)$, we have
\begin{equation*}
\begin{aligned}
\zeta_1(t)=te^{\frac{1}{2}\theta t^2}M_t^\alpha=\frac{M_t^\alpha}{\frac{1}{t}e^{-\frac{1}{2}\theta t^2}}\sim C_\alpha\frac{t^{\frac{1}{\alpha}+1}}{e^{-\frac{1}{2}\theta t^2}}\longrightarrow 0\quad(t\rightarrow\infty)
\end{aligned}
\end{equation*}
and 
\begin{equation*}
\begin{aligned}
\zeta_2(t)=\int_t^\infty(1+\theta s^2)e^{\frac{1}{2}\theta s^2}M_s^\alpha ds\sim C_\alpha\int_t^\infty (1+\theta s^2)e^{\frac{1}{2}\theta s^2}s^{\frac{1}{\alpha}}ds\longrightarrow 0 \quad(t\rightarrow\infty)
\end{aligned}
\end{equation*}
Now, we have proved that $\xi_t^\alpha$ converges to $\xi_\infty^\alpha$, as $t$ tends to infinity. And we consider the convergence in $L^p$ with $0<p<\alpha$.

For one thing, when $t\geq 0$ and $0<p<\alpha$, we have
$$
E(|\xi_t^\alpha|^p)\leq C_{p,\alpha}\left(\int_0^t s^\alpha e^{\frac{1}{2}\alpha \theta s^2}ds\right)^{\frac{p}{\alpha}}\leq \left(\int_0^\infty s^\alpha e^{\frac{1}{2}\alpha\theta s^2}ds\right)^{\frac{p}{\alpha}}<\infty,
$$
and
$$
E(|\xi_t^\alpha-\xi_\infty^\alpha|^p)\leq C_{p,\alpha}\left(\int_t^\infty s^\alpha e^{\frac{1}{2}\alpha\theta s^2}ds\right)^{\frac{p}{\alpha}}\sim C_{p,\alpha,\theta}e^{\frac{1}{2}p\theta t^2}t^{(1-\frac{1}{\alpha})p}\longrightarrow 0 \quad(t\rightarrow \infty).
$$

Thus, we can obtain the Lemma 4.1 by those convergences.
\end{proof}

\begin{lemma}
Let $\theta<0$, and define the funtion $I_\theta(t):\mathbb{R}^* \mapsto \mathbb{R}$:
$$
I_\theta(t):=te^{\frac{1}{2}\theta t^2}\int_0^t e^{-\frac{1}{2}\theta u^2}du+\frac{1}{\theta},
$$
we have $I_\theta(t)$ converges to zero as $t$ tends to infinity.
\end{lemma}

\begin{proof}
Given $\theta<0$, the function $t\mapsto I_\theta(t)$ has the convergence
\begin{equation}
\begin{aligned}
\lim_{t\rightarrow \infty}I_\theta(t)t^2 & = \lim_{t\rightarrow \infty}\frac{1}{\theta}t^2 e^{\frac{1}{2}\theta t^2}\left(e^{-\frac{1}{2}\theta t^2}+\theta t \int_0^t e^{-\frac{1}{2}\theta u^2}du\right)\\
&=\lim_{t\rightarrow \infty}\frac{1}{\theta}t^2 e^{\frac{1}{2}\theta t^2} \left(1-\int_0^t \theta ue^{-\frac{1}{2}\theta u^2}du+\theta  t \int_0^t e^{-\frac{1}{2}\theta u^2}du\right)\\
& = \lim_{t\rightarrow \infty} t^2 e^{\frac{1}{2}\theta t^2}\int_0^t e^{-\frac{1}{2}\theta u^2}(t-u)du\\
& =\lim_{t\rightarrow \infty}- \frac{1}{\frac{1}{t}\theta e^{-\frac{1}{2}\theta t^2}} \int_0^t e^{-\frac{1}{2}\theta u^2}du\\
& =\lim_{t\rightarrow \infty} \frac{1}{\theta^2 e^{-\frac{1}{2}\theta t^2}} e^{-\frac{1}{2}\theta t^2}=\frac{1}{\theta^2}
\end{aligned}
\end{equation}
Therefore, we can obtain
$$
|I_\theta(t)|\leq \frac{1}{\theta^2} \min\left\{1,\frac{1}{t^2}\right\}\longrightarrow0\quad(t\rightarrow\infty).
$$
\end{proof}

\begin{lemma}[X. Sun, L. Yan\cite{Yan2}]
When $\frac{1}{2}<\alpha \leq 2$, a random variable $\int_0^\infty J_\theta(s)dM_s^\alpha$ is well-defined in $L^p$ with $0<p<\alpha$ and 
\begin{equation}
\int_0^t J_\theta(s)dM_s^\alpha\longrightarrow \int_0^\infty J_\theta(s)dM_s^\alpha
\end{equation}
almost surely and in $L^p$ with $0<p<\alpha$, as $t$ tends to infinity where 
$$
J_\theta(t)=-\theta te^{\frac{1}{2}t^2}\int_0^t e^{-\frac{1}{2}\theta u^2}du-1.
$$
\end{lemma}

\begin{remark}
When we show the lemma above, as $t$ tends to infinity, for any $\theta<0,\frac{1}{2}<\alpha\leq 2$, denote
$$
\Psi_t:=\int_0^\infty I_\theta(s)dM_s^\alpha-\int_0^t I_\theta(s)dM_s^\alpha =\int_t^\infty I_\theta(s)dM_s^\alpha
$$
and $\Psi_t$ converges to zero almost surely. Recall the asymptotic structure of $t\mapsto I_\theta(t)$, we cannot prove the convergence in probability by (\ref{sec3-eq12}), but prove the lemma through the integration by part in Calculus.
\end{remark}

\begin{theorem}
Let $\theta<0$ and $0<\alpha\leq 2$. The process
$$
\Lambda_t: = e^{\frac{1}{2}\theta t^2}Y_t^\alpha,\quad t\geq 0
$$
converges to $\xi_\infty^\alpha -\frac{\nu}{\theta}$ almost surely and in $L^p$ with $0<p<\alpha$, as $t$ tends to infinity, where $Y_t^\alpha =\int_0^t (X_t^\alpha-X_s^\alpha)ds$.
\end{theorem}

\begin{proof}
Given $\theta<0$ and $0<\alpha\leq 2$, for all $t>0$ we have
\begin{equation*}
\begin{aligned}
\Lambda_t^\alpha&=e^{\frac{1}{2}\theta t^2}Y_t^\alpha=e^{\frac{1}{2}\theta t^2}\left(e^{-\frac{1}{2}\theta t^2}\int_0^t se^{\frac{1}{2}\theta s^2}dM_s^\alpha +\frac{\nu}{\theta}(1-e^{-\frac{1}{2}\theta t^2})\right)\\
&=\int_0^t  se^{\frac{1}{2}\theta s^2}dM_s^\alpha +\frac{\nu}{\theta}\left(e^{\frac{1}{2}\theta t^2}-1\right)\\
&=\xi_t^\alpha+\frac{\nu}{\theta}\left(e^{\frac{1}{2}\theta t^2}-1\right)
\end{aligned}
\end{equation*}
Lemma 4.1 implies that $\xi_t^\alpha$ converges to $\xi_\infty^\alpha$ almost surely and in $L^p$ with $0<p<\alpha$. Therefore, we can obtain the Theorem 4.1 by the Lemma 4.1 and the fact that $e^{\frac{1}{2}\theta t^2}\rightarrow 0$, as $t$ tends to infinity. 
\end{proof}

\begin{theorem}
Let $\theta<0$ and $0<\alpha\leq 2$. The process
$$
\Xi_t^\alpha :=te^{\frac{1}{2}\theta t^2}\int_0^t Y_t^\alpha dt,\quad t\geq 0
$$
converges to $\frac{1}{\theta}(\xi_\infty^\alpha-\frac{\nu}{\theta})$ almost surely and in $L^p$ with $0<p<\alpha$.
\end{theorem}

\begin{proof}
Given $\theta<0$, for all $t>0$, we have 
\begin{equation*}
\begin{aligned}
\Xi_t^\alpha&=te^{\frac{1}{2}\theta t^2}\int_0^t Y_t^\alpha dt=te^{\frac{1}{2}\theta t^2}\left(\int_0^t e^{-\frac{1}{2}\theta s^2}\int_0^s ue^{\frac{1}{2}\theta u^2}dM_u^\alpha ds+\int_0^t \frac{\nu}{\theta}(1-e^{-\frac{1}{2}\theta s^2})ds\right)\\
&=te^{\frac{1}{2}\theta t^2}\int_0^t e^{-\frac{1}{2}\theta s^2}\int_0^s ue^{\frac{1}{2}\theta u^2}dM_u^\alpha ds+\frac{\nu}{\theta}te^{\frac{1}{2}\theta t^2}\int_0^t (1-e^{-\frac{1}{2}\theta s^2})ds\\
&=te^{\frac{1}{2}\theta t^2}\int_0^t e^{-\frac{1}{2}\theta u^2}\xi_u^\alpha du +\frac{\nu}{\theta}t^2e^{\frac{1}{2}\theta t^2}-\frac{\nu}{\theta}te^{\frac{1}{2}\theta t^2}\int_0^t e^{-\frac{1}{2}\theta s^2}ds
\end{aligned}
\end{equation*}
and 
\begin{equation*}
\begin{aligned}
\int_0^t &e^{-\frac{1}{2}\theta u^2}(\xi_u^\alpha-\xi_\infty^\alpha)du=\int_0^t e^{-\frac{1}{2}\theta u^2}\left(\int_\infty^u se^{\frac{1}{2}\theta s^2}dM_s^\alpha\right)du\\
&=\int_0^t e^{-\frac{1}{2}\theta u^2}\left(\int_\infty^t se^{\frac{1}{2}\theta s^2}dM_s^\alpha+\int_t^u se^{\frac{1}{2}\theta s^2}dM_s^\alpha\right)du\\
&=-\int_0^t se^{\frac{1}{2}\theta s^2}\left(\int_0^s e^{-\frac{1}{2}\theta u^2}du\right)dM_s^\alpha-\left(\int_0^t e^{-\frac{1}{2}\theta u^2}du\right)\left(\int_t^\infty se^{\frac{1}{2}\theta s^2}dM_s^\alpha\right)\\
&\equiv -(\zeta_1^\alpha(t)+\zeta_2^\alpha(t)).
\end{aligned}
\end{equation*}
By the Lemma 4.1 and Lemma 4.2, for all $t>0$, we obtain
\begin{equation*}
\begin{aligned}
\Xi_t^\alpha-\frac{1}{\theta}\left(\xi_\infty^\alpha-\frac{\nu}{\theta}\right)&=te^{\frac{1}{2}\theta t^2}\int_0^t e^{-\frac{1}{2}\theta u^2}(\xi_u^\alpha-\xi_\infty^\alpha)du+\left(\xi_\infty^\alpha-\frac{\nu}{\theta}\right)I_\theta(t)+\frac{\nu}{\theta}t^2 e^{\frac{1}{2}\theta t^2}\\
&=-te^{\frac{1}{2}\theta t^2}[\zeta_1^\alpha(t)+\zeta_2^\alpha(t)]+\left(\xi_\infty^\alpha-\frac{\nu}{\theta}\right)I_\theta(t)+\frac{\nu}{\theta}t^2 e^{\frac{1}{2}\theta t^2}\\
\end{aligned}
\end{equation*}
as $t$ tends to infinity. 

Now, this will be done in two steps.

\textbf{(I)} We need to consider the convergence in probability of $\Xi_t^\alpha$ at first. By the Lemma 4.3, for any $t>0$, when $0<\alpha\leq 2$, $t$ tends to infinity, we have
\begin{equation*}
\begin{aligned}
te^{\frac{1}{2}\theta t^2}\zeta_1^\alpha(t)&=te^{\frac{1}{2}\theta t^2}\left(\int_0^t se^{\frac{1}{2}\theta s^2}\left(\int_0^s e^{-\frac{1}{2}\theta u^2}du\right)dM_s^\alpha\right)\\
&=te^{\frac{1}{2}\theta t^2}\left[\left(I_\theta(t)-\frac{1}{\theta}\right)M_t^\alpha-\int_0^tI'_\theta(s)M_s^\alpha ds\right]
\end{aligned}
\end{equation*}
where
\begin{equation*}
\begin{aligned}
I'_\theta(t)&=e^{\frac{1}{2}\theta t^2}\int_0^t e^{-\frac{1}{2}\theta u^2}du+\theta t^2e^{\frac{1}{2}\theta t^2}\int_0^t e^{-\frac{1}{2}\theta u^2}du+t\\
&=e^{\frac{1}{2}\theta t^2}\left(\int_0^t e^{-\frac{1}{2}\theta u^2}du+\theta t^2\int_0^t e^{-\frac{1}{2}\theta u^2}du+te^{-\frac{1}{2}\theta t^2}\right)\\
&= e^{\frac{1}{2}\theta t^2}\int_0^t e^{-\frac{1}{2}\theta u^2}(1+\theta t^2-t\theta u)du+te^{\frac{1}{2}\theta t^2},
\end{aligned}
\end{equation*}
According to the reference\cite{Yan2} by X. Sun and L. Yan, they applied the substitution $\frac{1}{2}\theta(t^2-u^2)=-x$ for the proof:
$$
e^{\frac{1}{2}\theta t^2}\int_0^t e^{-\frac{1}{2}\theta u^2}(1+\theta t^2-t\theta u)du \sim \frac{1}{\theta t}\int_0^{-\frac{1}{2}\theta t^2}e^{-x}(1-x)dx \sim \frac{1}{2}te^{\frac{1}{2}\theta t^2}
$$
Consequently, for all $\theta<0$, we have
$$
\lim_{t\rightarrow\infty}\frac{1}{t}e^{-\frac{1}{2}\theta t^2}I'_\theta(t)=\frac{3}{2}
$$
Considering the continuity of the function$t\mapsto I'_\theta(t)$, we can obtain
$$
|I'_\theta(t)|\leq \frac{3}{2} \min\left\{1,te^{\frac{1}{2}\theta t^2}\right\}.
$$
Therefore, it is obvious that
$$
te^{\frac{1}{2}\theta t^2}\zeta_1^\alpha(t)\longrightarrow 0 \quad (t\rightarrow \infty)
$$
almost surely.

Next we can consider the convergence in probability of the process $te^{\frac{1}{2}\theta t^2}\zeta_2^\alpha(t)$:
\begin{equation*}
\begin{aligned}
P\left(\sup_{t\geq T} \left|te^{\frac{1}{2}\theta t^2}\zeta_2^\alpha(t)\right|\geq \varepsilon \right)&\leq P\left(Te^{\frac{1}{2}\theta T^2}\int_0^T e^{-\frac{1}{2}\theta u^2}du\sup_{t\geq T} \left|\int_t^\infty se^{\frac{1}{2}\theta s^2 }dM_s^\alpha\right|\geq \varepsilon\right)\\
& \leq C_\alpha \varepsilon^{-\alpha}T^\alpha e^{\frac{1}{2}\alpha \theta T^2}\left(\int_0^T e^{-\frac{1}{2}\theta s^2}ds\right)^\alpha \left(\int_T^\infty s^\alpha e^{\frac{1}{2}\alpha \theta s^2}ds \right)\\
& \sim C_{\alpha,\theta}\varepsilon^{-\alpha} T^{\alpha-1}e^{\frac{1}{2}\alpha \theta T^2}\longrightarrow 0\quad (t\rightarrow\infty)
\end{aligned}
\end{equation*}
Therefore, combined with Lemma 4.2, the convergence $t^2e^{\frac{1}{2}\theta t^2}\rightarrow0$ and two convergences in probability above, we can obtain
\begin{equation}
\Xi_t^\alpha-\frac{1}{\theta}\left(\xi_\infty^\alpha-\frac{\nu}{\theta}\right)\longrightarrow 0
\end{equation}
almost surely.

\textbf{(II)} Now we consider the convergence in $L^p$ with $0<p<\alpha$上的收敛性 and we discuss these processes $te^{\frac{1}{2}\theta t^2}\zeta_1^\alpha(t)$ and $te^{\frac{1}{2}\theta t^2}\zeta_2^\alpha(t)$.
\begin{equation*}
\begin{aligned}
E\left|te^{\frac{1}{2}\theta t^2}\zeta_1^\alpha(t)\right|^p& \leq C_{\alpha,p}t^p e^{\frac{1}{2}\theta pt^2}E\left|\int_0^t se^{\frac{1}{2}\theta s^2}\left(\int_0^s e^{-\frac{1}{2}\theta u^2}du\right)dM_s^\alpha \right|^p\\
&\leq C_{\alpha,p}t^p e^{\frac{1}{2}\theta pt^2}\left(\int_0^t s^\alpha e^{\frac{1}{2}\alpha \theta s^2}\left(\int_0^s e^{-\frac{1}{2}\theta u^2}du\right)^\alpha\right)^{\frac{p}{\alpha}}\\
&\sim C_{\alpha,p,\theta}\left(t^{2\alpha-1}e^{\alpha\theta t^2}\int_0^t e^{-\frac{1}{2}\theta u^2}du\right)^{\frac{p}{\alpha}}\\
&\sim C_{\alpha,p,\theta} \left(t^{2\alpha-2}e^{\frac{1}{2}\alpha\theta t^2}\right)^{\frac{p}{\alpha}}\longrightarrow 0\quad(t\rightarrow \infty),
\end{aligned}
\end{equation*}
and
\begin{equation*}
\begin{aligned}
E\left|te^{\frac{1}{2}\theta t^2}\zeta_2^\alpha(t)\right|^p &\leq t^p e^{\frac{1}{2}\theta pt^2}\left(\int_0^t e^{-\frac{1}{2}\theta u^2}du\right)^p E\left|\int_t^\infty se^{\frac{1}{2}\theta s^2}dM_s^\alpha\right|^p\\
&\leq C_{p,\alpha,\theta} t^p e^{\frac{1}{2}\theta pt^2}\left(\int_0^t e^{-\frac{1}{2}\theta u^2}du\right)^p\left(\int_t^\infty s^\alpha e^{\frac{1}{2}\alpha \theta s^2}ds\right)^{\frac{p}{\alpha}}\\
&\sim C_{p,\alpha,\theta}t^{(1-\frac{1}{\alpha})p}e^{\frac{1}{2}p\theta t^2}\longrightarrow 0 \quad (t\rightarrow\infty)
\end{aligned}
\end{equation*}
Thus, we can get the convergence in probability and in $L^p(0<p<\alpha)$.
\end{proof}

%%%%%%%%%%%%%%%%%%%%%%%%%%%%%%%%%%%%%%%%
\begin{thebibliography}{0}

  \bibitem{Applebaum}
  D. Applebaum, {\it L\'evy Processes and Stochastic Calculus}. Cambridge University Press 2004.
  
  \bibitem{Benaim}
  M. Bena\"im, I. Ciotir and C.-E. Gauthier, Self-repelling diffusions via an infinite dimensional approach, {\em Stoch PDE: Anal. Comp.} {\bf 3} (2015), 506-530.
  
  \bibitem{Breiman}
  L. Breiman, A delicate law of the iterated logarithm for non-decreasing stable processes, {\it Ann. Math. Statist.} {\bf 39} (1968), 1818-1824.
  
  \bibitem{Cranston}
  M. Cranston and Y. Le Jan, Self-attracting diffusions: two case studies. {\it Math. Ann.} {\bf 303} (1995), 87-93.
  
  \bibitem{Cranston2}
  M. Cranston and T. S. Mountford, The strong law of large numbers for a Brownian polymer, {\it Ann. Probab.} {\bf 24} (1996), 1300-1323.
  
  \bibitem{Durrett}
  Durrett R. and Rogers L.C.G. (1991).  Asymptotic behavior of Brownian polymer. {\it Prob. Theory Rel. Fields} {\bf 92}, 337-349.
  
  \bibitem{Gan}
  Y. Gan and L. Yan, Least squares estimation for the linear self-repelling diffusion driven by fractional Brownian motion (in Chinese), {\em Sci. CHINA Math.} {\bf 48} (2018), 1143-1158.
  
  \bibitem{Gauthier}
  C.-E. Gauthier, Self attracting diffusions on a sphere and application to a periodic case, {\em Electron. Commun. Probab.} {\bf 21} (2016), No. 53, 1-12.
  
  \bibitem{Gine}
  E. Gin\'e and M. B. Marcus, The central limit theorem for stochastic integrals with respect to L\'evy processes, {\it Ann. Prob.} {\bf 11} (1983), 58-77.
  
  \bibitem{Herrmann}
  S. Herrmann and B. Roynette, Boundedness and convergence of some self-attracting diffusions, {\it Math. Ann.} {\bf 325} (2003), 81-96.
  
  \bibitem{Herrmann2}
  S. Herrmann and M. Scheutzow, Rate of convergence of some self-attracting diffusions, {\it Stochastic Process. Appl.} {\bf 111} (2004), 41-55.
  
  \bibitem{Kallenberg}
  O. Kallenberg, Some time change representations of stable integrals, via predictable transformations of local martingales. {\em Stoch. Process. Appl.} {\bf 40} (1992), 199-223.
  
  \bibitem{Kleptsyny}
  V. Kleptsyny and A. Kurtzmann, Ergodicity of self-attracting motion, {\em Electron. J. Probab.} {\bf 17} (2012), 1-37.
  
  \bibitem{Mountford}
  T. Mountford and P. Tarr\'es, An asymptotic result for Brownian polymers, {\it Ann. Inst. H. Poincar\'e Probab. Statist.}, {\bf 44} (2008), 29-46.
  
  \bibitem{Pemantle} 
  Pemantle R. (1988). Phase transition in reinforced random walk and RWRE on trees. {\it Ann. Probab.} {\bf 16}, 1229-1241.
  
  \bibitem{Protter}
  P. Protter, {\it Stochastic Integration and Differential Equations}, Springer, Berlin, Heidelberg and New York 2005.
  
  \bibitem{Rosinski}
  J. Rosinski and W. A. Woyczynski, On It\^o stochastic integration with respect to $p$-stable motion: inner clock, integrability of sample paths, double and multiple integrals, {\it Ann. Prob.} {\bf 14} (1986), 271-286.
  
  \bibitem{Samorodnitsky}
  G. Samorodnitsky, M. S. Taqqu, {\it Stable Non-Gaussian Random Processes}, Chapman $\And$ Hall 1994.
  
  \bibitem{Sato}
  K. Sato, {\it L\'evy Processes and Infinite Divisibility}, Cambridge University Press 1999.
  
  \bibitem{Yan2}
  X. Sun and L. Yan, Asymptotic Behavior on the Linear Self-Interacting diffusion driven by $\alpha$-Stable Motion, (2018).
  
  \bibitem{Chambeu}
  S. Chambeu and A. Kurtzmann, Some particular self-interacting diffusions:
  Ergodic behaviour and almost sure convergence, {\em Bernoulli} {\bf 17} (2011), 1248-1267.
  
  \bibitem{Werner}
  B. T\'oth and W. Werner, The true self-repelling motion, {\it Probab. Theory Relat. Fields} {\bf 111} (1998), 375-452.
  
  \bibitem{Yan}
  L. Yan, Y. Sun and Y. Lu, On the linear fractional self-attracting diffusion, {\it J. Theort. Probab.} {\bf 21} (2008), 502-516.
  \end{thebibliography}
\end{document}
